\documentclass[12pt,onecolumn]{article}
\usepackage[utf8]{inputenc} % UTF8 input encoding
\usepackage[T2A]{fontenc}   % T2A font encoding for Cyrillic script
\usepackage[russian]{babel} % Russian language support
\usepackage{listings}
\usepackage{float}
\usepackage{mathtools}
\usepackage{longtable}
\everymath{\displaystyle}
\usepackage{listings} 
\usepackage[usenames]{color}
\usepackage[html]{xcolor}
\usepackage{framed}
\usepackage{csquotes}
\usepackage{geometry}

\geometry{
  a4paper,
  top=25mm, 
  right=5mm, 
  bottom=25mm, 
  left=5mm
}

\begin{document}
\setcounter{tocdepth}{4}
\begin{center}
    Федеральное государственное автономное образовательное учреждение высшего образования "Национальный Исследовательский Университет ИТМО"\\ 
    Мегафакультет Компьютерных Технологий и Управления\\
    Факультет Программной Инженерии и Компьютерной Техники \\
    \includegraphics[scale=0.3]{image/itmo.jpg} % нужно закинуть картинку логтипа в папку с отчетом
\end{center}
\vspace{1cm}


\begin{center}
    \textbf{Лабораторная работа №1}\\
    по дисциплине\\
    \textbf{'Архитектура программных систем'}\\
\end{center}

\vspace{2cm}

\begin{flushright}
  Выполнил Студент  группы P33102\\
  \textbf{Лапин Алексей Александрович}\\
  Преподаватель: \\
  \textbf{Перл Иван Андреевич}\\
\end{flushright}

\vspace{6cm}
\begin{center}
    г. Санкт-Петербург\\
    2023г.
\end{center}

\newpage
\tableofcontents
\newpage

\section{Текст задания}
Выбрать любую реально существующую систему и описать её в терминах UML. Желательно, чтобы система была не полностью информационной, но опиралась на информационную систему как показано в примере на лекции (Point of sale). Необходимо описать границы системы на разных уровнях, а также описать сценарии использования для нескольких Акторов.\\
\textbf{Отчёт по работе должен содержать:}
\begin{enumerate}
  \item Титульный лист с указанием автора и номера группы
  \item Само задание
  \item Описание рассматриваемой системы с требованиями к ней
  \item Формальное описание системы с необходимым количеством UML диаграмм
  \item Словесное описание сценариев сценариев использование для рассматриваемых акторов
\end{enumerate}
\section{Описание рассматриваемой системы с требованиями к ней}
Онлайн магазин для тренеров по продаже покемонов и всего что с ними связано.\\
Возможности:
\begin{itemize}
  \item Ввести свои характеристи (стиль игры, уровень, своих покемонов) и получить рекомендованные к покупке покемоны
  \item Положить понравившихся покемонов в корзину
  \item Оформить заказ
  \item Отслеживать статус заказа
  \item Посмотреть покупки
\end{itemize}
\subsection{Функциональные требования (FR)}
\textbf{Требования владельца сайта:}
\begin{enumerate}
  \item Возможность добавлять новых покемонов в каталог
  \item Возможность удалять покемонов из каталога
  \item Возможность изменять характеристики покемонов в каталоге
  \item Возможность видеть заказы
  \item Возможность изменять статус заказа
\end{enumerate}
\textbf{Требования пользователя:}
\begin{enumerate}
  \item Возможность регистрации
  \item Возможность авторизации
  \item Возможность просмотра каталога
  \item Возможность сортировки каталога
  \item Возможность фильтрации каталога
  \item Возможность просмотра информации о выбранном покемоне
  \item Возможность добавления покемонов в корзину
  \item Возможность оформления заказа (оплата, доставка)
  \item Возможность отслеживания статуса заказа
  \item Возможность просмотра истории заказов
  \item Возможность ввести свои характеристики и получить рекомендованные к покупке покемоны
  \item Возможность изменения своих характеристик
\end{enumerate}
\subsection{Нефункциональные требования (NR)}
\begin{enumerate}
  \item Сайт должен корректно работать во всех современных браузерах. Такие как:
  Google Chrome, Mozilla Firefox, Microsoft Edge, Яндекс.Браузер.
  \item Сайт должен быть адаптирован под мобильные устройства.
  \item Backend должен быть написан на языке программирования Java с фреймворком JavaSpring.
  \item Frontend должен быть написан на языке программирования JavaScript с использованием библиотеки React.
  \item Хранение данных должно быть реализовано с помощью СУБД PostgreSQL.
  \item Сайт должен быть развернут на helios
  \item Сайт должен быть защищён от SQL инъекций
\end{enumerate}
\section{Формальное описание системы с необходимым количеством UML диаграмм}

\section{Словесное описание сценариев сценариев использование для рассматриваемых акторов}
\begin{longtable}{|l|l|}
  \hline
  Название сценария & Регистрация                                             \\ \hline
  \endfirsthead
  %
  \endhead
  %
  Краткое описание  & Пользователь хочет создать свой личный кабинет на сайте \\ \hline
  Акторы            & Пользователь                                            \\ \hline
  Предусловия       & Пользователь не зарегистрирован на сайте                \\ \hline
  Основной поток &
    \begin{tabular}[c]{@{}l@{}}1. Пользователь нажимает на кнопку "Sign up"\\2. Система отображает форму регистрации\\ 3. Пользователь вводит свои данные\\ 4. Пользователь нажимает на кнопку "Sign up"\\ 5. Система регистрирует пользователя\\ 6. Система перенаправляет пользователя на главную страницу\\ 7. Система отображает сообщение об успешной регистрации\end{tabular} \\ \hline
  Постусловия       & Пользователь может войти в свой личный кабинет          \\ \hline
  \end{longtable}

\begin{longtable}{|l|l|}
  Название сценария & 
  Краткое описание  & 

\end{longtable}
\end{document}
